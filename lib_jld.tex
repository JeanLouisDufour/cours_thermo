%%%% \documentclass[11pt]{article} % use larger type; default would be 10pt

\usepackage[utf8]{inputenc} % set input encoding (not needed with XeLaTeX)

%%% Examples of Article customizations
% These packages are optional, depending whether you want the features they provide.
% See the LaTeX Companion or other references for full information.

%%% PAGE DIMENSIONS
\usepackage{geometry} % to change the page dimensions
\geometry{a4paper} % or letterpaper (US) or a5paper or....
% \geometry{margin=2in} % for example, change the margins to 2 inches all round
% \geometry{landscape} % set up the page for landscape
%   read geometry.pdf for detailed page layout information

\usepackage{graphicx} % support the \includegraphics command and options

% \usepackage[parfill]{parskip} % Activate to begin paragraphs with an empty line rather than an indent

%%% PACKAGES
\usepackage{booktabs} % for much better looking tables
\usepackage{array} % for better arrays (eg matrices) in maths
\usepackage{paralist} % very flexible & customisable lists (eg. enumerate/itemize, etc.)
\usepackage{verbatim} % adds environment for commenting out blocks of text & for better verbatim
\usepackage{subfig} % make it possible to include more than one captioned figure/table in a single float
% These packages are all incorporated in the memoir class to one degree or another...

\usepackage{tikz}
%% pour le diag P-V
\usetikzlibrary{decorations.markings, arrows, arrows.meta}
\tikzset{
    midar/.style 2 args={
        very thick,
        decoration={name=markings,
        mark=at position .55 with {\arrow{latex}},
        mark=at position 0 with {\fill circle (2pt);},
        mark=at position 1 with {\fill circle (2pt);}}
        ,postaction=decorate,
    },
}
%% pour la machine
\usetikzlibrary{shapes,arrows}
\tikzset{%
pics/.cd,
nodea/.style args={#1#2#3}{
  code={\node[minimum height=2cm] (#3) {\color{#1}#2};
       \draw[thick] (#3.south west) -| (#3.north east)--(#3.north west);
  }
},
%pics/.cd,
nodeb/.style args={#1#2#3}{
  code={\node[minimum height=2cm] (#3) {\color{#1}#2};
       \draw[thick] (#3.south east) -| (#3.north west)--(#3.north east);
  }
},
%pics/.cd,
nodec/.style args={#1#2#3}{
  code={\node[draw,thick,shape=circle,inner sep=1cm] (#3) {\color{#1}#2};
  }
},
}
%% pour http://texample.net/media/tikz/examples/TEX/is-lm-diagram.tex
\usetikzlibrary{arrows,calc}
\usepackage{relsize}
\newcommand\LM{\ensuremath{\mathit{LM}}}
\newcommand\IS{\ensuremath{\mathit{IS}}}
%% pour...

%%% HEADERS & FOOTERS
\usepackage{fancyhdr} % This should be set AFTER setting up the page geometry
\pagestyle{fancy} % options: empty , plain , fancy
\renewcommand{\headrulewidth}{0pt} % customise the layout...
\lhead{}\chead{}\rhead{}
\lfoot{}\cfoot{\thepage}\rfoot{}

%%% SECTION TITLE APPEARANCE
\usepackage{sectsty}
\allsectionsfont{\sffamily\mdseries\upshape} % (See the fntguide.pdf for font help)
% (This matches ConTeXt defaults)

%%% ToC (table of contents) APPEARANCE
\usepackage[nottoc,notlof,notlot]{tocbibind} % Put the bibliography in the ToC
\usepackage[titles,subfigure]{tocloft} % Alter the style of the Table of Contents
\renewcommand{\cftsecfont}{\rmfamily\mdseries\upshape}
\renewcommand{\cftsecpagefont}{\rmfamily\mdseries\upshape} % No bold!

%%% The "real" document content comes below...
